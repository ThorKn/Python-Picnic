\documentclass[]{article}

% ------
% Fonts and typesetting settings
\usepackage[sc]{mathpazo}
\usepackage[T1]{fontenc}
\linespread{1.05} % Palatino needs more space between lines
\usepackage{microtype}
\usepackage{amsmath}

% ------
% Page layout
\usepackage[hmarginratio=1:1,top=30mm,bottom=30mm,columnsep=20pt]{geometry}
\usepackage[font=it]{caption}
\usepackage{paralist}
%\usepackage{multicol}

\usepackage{graphicx}
\usepackage{color}
\usepackage{tabularx}

%\usepackage{wrapfig}

\usepackage{capt-of}

% \usepackage[ngerman]{babel}

% ------
% Lettrines
\usepackage{lettrine}

% ------
% Abstract
\usepackage{abstract}
	\renewcommand{\abstractnamefont}{\normalfont\bfseries}
	\renewcommand{\abstracttextfont}{\normalfont\small\itshape}


% ------
% Titling (section/subsection)
\usepackage{titlesec}
\renewcommand\thesection{\Roman{section}}
\titleformat{\section}[block]{\large\scshape\centering}{\thesection.}{1em}{}


% ------
% Header/footer
\usepackage{fancyhdr}
	\pagestyle{fancy}
	\fancyhead{}
	\fancyfoot{}
	\fancyhead[C]{Post-quantum cryptography $\bullet$ Winter 2018/19 $\bullet$ RheinMain University of Applied Sciences}
	\fancyfoot[RO,LE]{\thepage}


% ------
% Maketitle metadata
\title{\vspace{-7mm}%
	\fontsize{24pt}{10pt}\selectfont
	\textbf{Re-implementation of the Picnic signaturescheme in Python}
	}	
\author{%
	\large
	\textsc{Thorsten Knoll} \\[2mm]
	\normalsize	info@thorstenknoll.de \\[2mm]
	\normalsize March, 2019
	%\vspace{-5mm}
	}
\date{}

\usepackage[utf8]{inputenc}

\usepackage{hyperref}

%%%%%%%%%%%%%%%%%%%%%%%%
\begin{document}

\maketitle
\thispagestyle{fancy}

%\begin{abstract}
%\noindent Abstract goes here
%\end{abstract}

\section{Introduction}
\subsection{Post-quantum cryptography, NIST and Picnic}
Quantum computers are experiencing fast development and seem to be available within a timeframe of the next few decades. One of their properties will be to break huge parts of modern cryptography. Especially the discrete logarithm and prime-factorisation loose their trapdoor funcionality in regards to the efficient quantum algorithms from Grover and Shor. Therefore the need for new cryptographic algorithms arises, beeing save in regards to the availability of quantum computers. This field of research goes with the name "Post-quantum cryptography" (PQC). The american National Institute of Standards and Technology (NIST) called out a challenge to find the next PQC standards. This challenge is in round two of three at the time of writing this document. 69 submissions from round one were reduced to 26 candidates in round two of the challenge. These 26 candidates got announced by NIST not long ago at the end of january 2019. One of the submissions surviving the first round is the Picnic signaturescheme.\\ \\
NIST PQC page: \url{https://csrc.nist.gov/Projects/Post-Quantum-Cryptography} 
\subsection{Evaluation and reviews}
69 submissions is quite a huge amount in terms of reviewing and evaluating them. Additionaly PQC is a new field of research with a history not much longer than a decade. Cryptographers (and -analysts) are working intense on creating, evaluating, reviewing and breaking PQC algorithms. The state of PQC is that every litte step helps towards the goal of having secure PCQ algorithms. At RheinMain University of Applied Sciences (HSRM) the Master Students in Computer Science found together in a research course to participate in this process. This re-implementation is part of this efford.
\subsection{Goals: Learning Picnic (and LowMC)}
The sole purpose of this re-implementation is learning and understanding the Picnic algorithm and it's underlying zero-knowledge proof system. The goals are to provide an "easy to read and understand" codebase in a high-level language (Python) and make it a little easier to follow and learn the designprinciples of Picnic. The execution of Python-Picnic is awefull slow compared to the reference implementations in C. It is very unlikely that this will ever reach a usable state for production. Additionaly the code is not reviewed by anyone but the author and is surely not secure for productive use. But it may help to understand Picnic.\\ \\
On the way to understand Picnic, another algorithm must be understood too. That is the LowMC blockcipher. There are two main parts in Picnic, where LowMC plays a major role. Firstly the derivation of the public key is a LowMC encryption of the Picnic private key. Secondly each Picnic zero-knowledge round is a modified LowMC encryption, fitted into a Multi-Party-Computation (MPC) scheme. In the following description of the implementations we start with LowMC for that reason.
\subsection{Reference documents and implementations by the Picnic Team}
The original publications and reference implementations (in C) are available at: \\ \\
Microsoft Picnic projectpage: \url{https://www.microsoft.com/en-us/research/project/picnic/}\\
Microsoft Picnic Github: \url{https://github.com/Microsoft/Picnic}
\subsection{Re-implementations in Python}
These re-implementations in Python follow the original MIT Licenses and are public available at:\\ \\
LowMC in Python: \url{https://github.com/ThorKn/Python-LowMC}\\
Picnic in Python: \url{https://github.com/ThorKn/Python-Picnic}\\ \\
Installation and usage instructions are given inside the projects \texttt{README} files. This \texttt{PDF} can be found inside the \texttt{Python-Picnic} project in the \texttt{docs} folder.
\section{LowMC}
\subsection{Idea and overview}
Picnic makes intense use of the LowMC algorithm. Therefore we start with LowMC before hoping over to Picnic. A standalone LowMC Python-implementation got build as a starting point to understand Picnic. That is why there is a own Github project under the name \texttt{"Python-LowMC"}.\\ \\ 
LowMC is a blockcipher in a roundbased construction scheme like many other blockciphers. The name is an abbreviation for "Low multiplicative complexity". XOR in GF2 is a linear operation (ADD), while the multiplication (AND) in GF2 is a non-linear operation (Figure \ref{fig:gates}). LowMC tries to keep the count of AND operations as low as possible while still maintaining a given security level (L1, L3, L5). Additionaly LowMC is also designed to keep the AND-depth low. That means most AND operations could be done in parallel.\\
\begin{figure}[htbp]
\center
\includegraphics[width=0.35\textwidth]{pics/gates.pdf}
\caption{Linear and non-linear operations}
\label{fig:gates}
\end{figure}\\
Figure \ref{fig:lowmcscheme} shows the roundbased scheme of LowMC. Only the sbox part of the algorithm uses multiplications (ANDs). The other parts strictly contain only linear operations. From a mathematical point of view the XOR is a bijective function and the multiplications (ANDs) in GF2 are not bijective. Therefore the sboxes are the only not bijective part in LowMC. That is the reason for choosing LowMC in Picnic and it keeps the signature lengths smaller than with other blockciphers. We'll see how this is achieved with the Picnic implementation later on. 
\begin{figure}[htbp]
\center
\includegraphics[width=0.5\textwidth]{pics/lowmc_scheme.pdf}
\caption{LowMC scheme}
\label{fig:lowmcscheme}
\end{figure}
\subsection{Pre-calculated constants}
The python program \texttt{generator.py} generates the files for all security levels with the following pre-calculated constants:
\begin{itemize}
\item{Linear layer matrices}
\item{Round constants}
\item{Roundkey matrices}
\end{itemize}
The generation of the constants-files is not mandatory for usage as the project contains them "ready to use". 
\subsection{Private functions}
\subsubsection{\texttt{\_\_apply\_sbox()}}
Sboxes are the only parts in LowMC, that contain multiplications in GF2 (ANDs). A fixed number of sboxes is applied to the state. Each sbox substitutes 3 Bit of the original state through a fixed substitution scheme. Let $a$, $b$ and $c$ be three bits in the state. Then $a'$, $b'$ and $c'$ get computed by:
\newpage
\[a' = a \oplus b \cdot c\]
\[b' = a \oplus b \oplus a \cdot c\]
\[c' = a \oplus b \oplus c \oplus a \cdot b\]
Where $\oplus$ is XOR and $\cdot$ is AND. So each 3-Bit sbox contains exactly 3 multiplications (ANDs). The security level (L1, L3, L5) defines the number of 3-Bit sboxes ($n$) per LowMC-round as shown in figure \ref{fig:sboxes}. The surplus Bits in the state get no substitution, shown on the right of the figure. The total number of multiplications for a complete encryption in LowMC sums up to $3 * n * rounds$. In example for security level L1 this calculates to $3 * 10 * 20 = 600$ ANDs.\\
\begin{figure}[htbp]
\center
\includegraphics[width=0.5\textwidth]{pics/sboxes.pdf}
\caption{Sboxes per LowMC round}
\label{fig:sboxes}
\end{figure}\\
The function applies the sboxes to a state in memory and has no parameters and no returns. The actual state is stored in the private variable \texttt{self.\_\_state}. The sboxes are stored as a list (you can imagine it as a lookup-table) in the private variable \texttt{self.\_\_sbox}.
\subsubsection{\texttt{\_\_multiply\_with\_lin\_mat(round)}}
The state gets matrix multiplied with the constant and pre-calculated linear matrix. This contains only linear operations.
\subsubsection{\texttt{Add roundconstant}}
This needs no seperate function. It is a one-line operation with XOR on the state and therefore contains only linear operations.
\subsubsection{\texttt{\_\_key\_addition(round)}}
For XOR'ing the roundkey to the state the roundkey must be derived first. This is done by a matrix multiplication of the private key and the constant, pre-calculated roundkey-matrix. Again this only contains linear operations. 
\subsubsection{\texttt{Decryption functions}}
The decryption works pretty much the whole encryption way backwards. For the matrix multiplications their inverse matrices are needed. They get calculated within the constructor of the LowMC class and stored in seperate, private variables. The neccessary functions are named like the ones for encryption but with a \texttt{\_inv} appended to their names. The same rule applies for the names of their private variables.
\subsection{Public functions (API)}
\subsubsection{\texttt{LowMC(Security level) - Constructor}}
Constrcuts an object of LowMC with the parameters regarding to the security level. The following security levels are available and shall be given to the constructor as strings: "picnic-L1", "picnic-L3" and "picnic-L5". The fitting file with the constants must be in the project directory and gets read (see Pre-calculated constants). The constants from the file get stored in the private variables \texttt{self.\_\_lin\_layer}, \texttt{self.\_\_round\_consts} and \texttt{self.\_\_round\_key\_mats}.
\subsubsection{\texttt{generate\_priv\_key()}}
Generates a private key of the length specified within the security level. The private key is stored in the private variable \texttt{self.\_\_priv\_key}. The CSPRNG from the python package \texttt{os} is used (\texttt{os.urandom(bytelength)}).
\subsubsection{\texttt{set\_priv\_key(priv\_key)}}
Instead of generating the private key, it can also be set by giving a bytearry to this function. The bytearray must match the specified keylength from the security level.
\subsubsection{\texttt{encrypt(plaintext)}}
Encrypts a plaintext and returns the ciphertext. The plaintext length must match the specified blocksize length (security level) and must be given as a bytearray to the function. The ciphertext is returned as a bytearray of the same length. Before using this function a private key must be set (or generated).
\subsubsection{\texttt{decrypt(ciphertext)}}
Decrypts a ciphtertext and returns the plaintext. The ciphertext length must match the specified blocksize length (security level) and must be given as a bytearray to the function. The plaintext is returned as a bytearray of the same length. Before using this function a private key must be set (or generated).
\subsection{Testvectors}
The repository contains the python file \texttt{test\_lowmc.py}. One can simply run this and nine different testvectors get executed on the implementation. Three vectors for each security level. This testfile is also a good starting point to see how the implementation can be used.
\subsection{Prerequisites}
The code is tested with Ubuntu 16.04 LTS and Python3.6. The package "BitVector" for python is required. It is recommended to use a virtual environment for Python, like \texttt{virtualenv}. In (very) short lines:\\
\texttt{virtualenv --p /usr/bin/python3.6 myvenv}\\
\texttt{source /path\_to\_myvenv/bin/activate}\\
\texttt{<myvenv>pip install BitVector}\\
\texttt{<myvenv>python test\_lowmc.py}\\
\section{Picnic}
\subsection{Components}
Picnic is a Post-quantum signature scheme that does not rely on hard number theoretical problems like discrete logarithm or prime-factorization. Instead Picnice embedds symmetric cryptographic primitves into a zero-knowledge proof system. The components of Picnic are:
\begin{itemize}
\item{A blockcipher (LowMC)}
\item{A Hashfunktion (SHA3-SHAKE)}
\item{A Zero-knowledge proof system (ZKB++)}
\end{itemize}
LowMC is discussed earlier in this document. It is a parametric blockcipher algorithm that simulates a gatebased (XOR, AND in GF2) electrical circuit from input to output with low AND-gate counts. The Python-LowMC implementation will be used for Picnic. SHA3-SHAKE will not be discussed in detail in this document. It is a NIST-standardized Hashalgorithm, based on the Spongeconstruction with an arbitrary output length. In Picnic SHA3-SHAKE is used as a Hashfunction as well as a Key-Derivation-Function (KDF). Python-Picnic uses the Python library \texttt{hashlib} whereever SHA3-SHAKE is needed\\ \\
That leaves the zero-knowledge proof system ZKB++ to explain. The following descriptions are on a very abstract level and will not explain every detail. Instead we'll focus on getting a broad overview of the functionality to learn the key points of the implementation. Most of the following examples are based on the documentations and presentations of the Picnic research team \footnote{\url{https://asiacrypt.iacr.org/2018/files/SLIDES/TUESDAY/Z411/post\%20quantum\%20signatures\%20-\%20asiacrypt18v2.pdf}}.
\subsection{Proof of knowledge: The $\Sigma$ - Protocol}
Zero-knowledge proofs are based on the $\Sigma$-protocol, but with the twist to transmitt no parts of the underlying secret anywhere in the communication. The basic $\Sigma$-Protocol is a communication scheme between two parties as in figure \ref{fig:sigma}. A prover wants to convince a verifier about the knowledge of a secret. 
\begin{figure}[htbp]
\center
\includegraphics[width=0.5\textwidth]{pics/sigma.pdf}
\caption{$\Sigma$-Protocol scheme}
\label{fig:sigma}
\end{figure}\\
To prove this knowledge, a 3-way communication is held. A classic example for the $\Sigma$-Protocol is the Schnorr protocol. It is based on the discrete logarithm. The prover knows a secret $x$ so that $y=g^{x}$ with $g$ beeing a generator in a cyclic group $G_{p}$ with the order $p$. The messages in the Schnorr protocol are:
\begin{itemize}
\item{Commitment: Prover chooses a random $r$ and commits $t = g^{r}$ and $y$.}
\item{Challenge: Verifier chooses a random $c$ and sends it to the prover.}
\item{Response: Prover sends $s = r + cx$.}
\end{itemize}
The verifier accepts, if $ty^{c} = g^{s}$. Because:
\[ty^{c} = g^{r}y^{c} = g^{r}g^{cx} = g^{r + cx} = g^{s}\]
Inserted into the sequence diagramm above, this communication would look like figure \ref{fig:schnorr}.
\begin{figure}[htbp]
\center
\includegraphics[width=0.5\textwidth]{pics/schnorr.pdf}
\caption{Schnorr protocol}
\label{fig:schnorr}
\end{figure}
\subsection{LowMC in ZKB++}
In the next step towards ZKB++ we'll look into how LowMC works as the One-way-function inside a $\Sigma$-protocol.\\ \\
Imagine LowMC as a electronic circuit of XOR and AND gates as in the example in figure \ref{fig:circuit}. There are inputs ($x_{1..8}$) and outputs ($y_{1..6}$) from the circuit. The function of this LowMC circuit could be $f_{LowMC}(x) = y$. For given inputs the outputs can be calculated efficiently. For given outputs there is no efficient algorithm to determine the inputs. That is a hard to reverse function.\\
\begin{figure}[htbp]
\center
\includegraphics[width=0.69\textwidth]{pics/circuit.pdf}
\caption{A LowMC circuit as a One-way-function}
\label{fig:circuit}
\end{figure}\\
Thinking in terms of the $\Sigma$-protocol we can define the following sentence:
\begin{center}
\textit{A prover has knowledge of a secret key (inputs $x$) that computes with\\
the LowMC cicruit (One-way-function) to a public key (outputs $y$).}
\end{center}
\subsection{MPC and the zero-knowledge proof} 
The zero-knowledge proof system ZKB++ is based on Multi-Party-Computation (MPC). MPC means that the prover from the $\Sigma$-protocol is splitted into different players. For ZKB++ the number of players is fixed to three. To get a grip of how MPC and the zero-knowledge property works, we'll look into an example from Melissa Chase from the Picnic research team (figure \ref{fig:mpc}. She presented this example at RealWorldCrypto 2018 and the talk is public available as a video\footnote{\url{https://www.youtube.com/watch?v=_J9ESIy8D2o}}.\\ \\
A prover knows the secrets $a$ and $b$. The function $f$ shall be the very simple circuit $c=a\oplus c$, where $\oplus$ is an XOR. Let $H$ be a cryptographic secure Hashfunction (i.e. SHA3-SHAKE). This example is designed for two players, instead of the three players in ZKB++.\\
\begin{figure}[htbp]
\center
\includegraphics[width=1.0\textwidth]{pics/mpc2player.pdf}
\caption{MPC with two players}
\label{fig:mpc}
\end{figure}\\
It is easy to recognize the $\Sigma$-protcol in this MPC by the message scheme of "commitment", "challenge" and "response". The differences are that now two players are involved on the prover side. Both have their input- and output shares and the challenge determines which players input shares get revealed. The verifier can then check if the revealed player had a valid input share of the secrets $a$ and $b$ and therefore is convinced about the provers knowledge of the secrets by a probability of 50\% (one of two players). This can be repeated $n$ times till the wanted probability (defined by the security level) is reached. The probability calculates to $p = (1/2)^{n}$. As the function $f$ is assumed to be "hard to reverse", nothing about the secrets $a$ and $b$ got learned. That is the zero-knowledge part of the proof.\\ \\
Some remarks about the differences between this example and the MPC in Picnic need to be noted:
\begin{itemize}
\item{This works because of $(a_1 \oplus a_2) \oplus (b_1 \oplus b_2) = (a_1 \oplus b_1) \oplus (a_2 \oplus b_2) = c_1 \oplus c_2 = c$}
\item{LowMC has AND gates. This makes dividing the shares more complicated.}
\item{ZKB++ has three players instead of two. The secrets have to be shared by three.}
\item{The cheating probability increases to $2/3$. Confidence probability is $p=(1/3)^n$.}
\item{Challenges request one of three players to be revealed with $e \in \{0,1,2\}$.}
\item{The lowest security level in Picnic has $n = 219$ MPC rounds.}
\end{itemize} 
\subsection{MPC in the head and Random oracle models}
The zero-knowledge proof we've discussed so far is interactive between the prover and the verifier. The communication in the $\Sigma$-protocol defines the verifier as a secure source of randomness for the challenges. The prover must not be able to guess in advance what the challenge will be.\\ \\ 
For a signaturescheme it is wanted to calculate the proof without the need for an "external" source of randomness. The signatures shall be created and verified on a single machine without the counterpart of the verifier in the communication. This is called "MPC in the head", reduces the proof to a non-interactive version and is used in ZKB++. It wouldn't be a secure idea to let the prover choose the challenges on his own. The prover then could simply pick the challenge that favours him. A proveable secure solution to this is to use a "Random Oracle Model" (ROM) on the provers side to generate the challenges. In ZKB++ this is done with the cryptographic secure hashfunction SHA3-SHAKE. The output of SHA3-SAHKE is the source of randomness under the premise of the ROM and fulfills all the needed properties for secure random numbers.\\ \\
The concrete implementation looks like this:\\
The round-commitments contain randomness (input shares) and are calculated and commited before the start of the protocol. To get the challenges, the commitments from each MPC round gets hashed. In the example above (figure \ref{fig:mpc}) this would be the hash of $h_1, h_2, c_1, c_2$. The challenges then are extracted from this hash bitwise (figure \ref{fig:challenges}). Each two bits from the hash are transformed to a single challenge $e \in \{0,1,2\}$ for a three party MPC. And there we see the reason for using a hashfunction with arbitrary output lengths (SHA3-SHAKE). The needed length of this hash depends on the number of MPC rounds $n$ (security level L1: $n=219$) as every MPC round has it's own challenge.\\
\begin{figure}[htbp]
\center
\includegraphics[width=0.7\textwidth]{pics/challenges.pdf}
\caption{Hashes to challenges}
\label{fig:challenges}
\end{figure}\\ \\
Picnic has two different versions of MPC in the head defined. The first one is the Fiat-Shamir (FS) transformation, which is based on the described ROM but might not be quantum save. The second option is the Unruh (UR) transformation, which is based on a Quantum Random Oracle Model (QROM). The re-implementation Python-Picnic can only handle the FS transformation so far. 
\subsection{Signing a message: The Picnic loops}
All the described parts taken together assemble to the Picnic signaturescheme. Picnic consists mainly of two big loops, shown in figure \ref{fig:mainloops} for creating a signature (signing a message).
\begin{figure}[htbp]
\center
\includegraphics[width=0.5\textwidth]{pics/mainloops.pdf}
\caption{Signing a message: Picnic loops}
\label{fig:mainloops}
\end{figure}\\
\subsection{Keys, Message and Signature}
Picnic works with keypairs for signing and verifying as every other public key cryptographic algorithm. In figure \ref{fig:mainloops} the parts are marked, where the keys and the message to sign is used. The generation of the keypair is done with the LowMC encryption. A random plaintext message is needed to do this. The lengths of the keys and the plaintext are defined through the security level. Therefore the \textbf{key generation} is:
\begin{itemize}
\item{Pick a random private key $sk$ (secret key).}
\item{Pick a random plaintext message $p$ (Not to confound with the message to sign).}
\item{Compute the encryption of $p$ with $sk$: $C = LowMC(sk, p)$.}
\item{The public key is the tuple $pk = (C,p) = (LowMC(sk,p), p)$.}
\end{itemize}
The \textbf{message $M$} to be signed must be an arbitrary bytearray of the length $1 \leq \vert M  \vert \leq 2^{55}$. The messsage takes account into the creation of the challenges. In the description of "MPC in the head" above, this was not explained to keep things as simple as possible. But in the Picnic signaturescheme the challenges depend on the commitments, the output shares \textbf{and} the message $M$.\\ \\
The Picnic signing function is defined as:
\[signature\text{ }\delta = picnic\_sign(sk, pk, M)\]
The resulting \textbf{signature $\delta$} can be understood as a transcript of the complete communication in the MPC protocol. The signature in Picnic includes:
\begin{itemize}
\item{The challenges for all MPC rounds.}
\item{The player revelations according to the challenges (transcripts, seeds and input shares). Remark: Only one player gets revealed per MPC round.}
\item{The commitments of the player according to the challenges. Remark: Only one players commitment gets revealed per MPC round.}
\end{itemize}
\textbf{LowMc and the signature sizes:}\\
At the beginning of the LowMC description it was said, that LowMC was chosen because it reduces the signature sizes in Picnic, compared to other blockcipher algorithms. The argumentation was, that the low count of AND gates does this. In figure \ref{fig:mainloops} can be seen that the transcripts of the LowMC rounds are done in the sbox-part of the algorithm. And only the calculations from the AND gates get recorded as a transcript. That is the point where Picnic takes advantage of LowMC to minimize the size of the signatures.
\subsection{Types and variables in the implementation}
With the given explanations about the Picnic algorithm it should be not that hard to understand the Python sourcecode. The functions mostly map directly to the explanations. Instead of describing all functions in detail, an affiliation of the code-variables to the examples above will be given. We'll start with the public variables in table \ref{tab:secparams} defined through the security level of Picnic. If Python-Picnic would support more security levels (only L1 at the moment), the constructor would set these variables according to them. Supporting more security levels is implemented in Python-LowMC and it can be seen there, as an example how to implement this feature.\newpage
\begin{table}[htbp]
\begin{center}
\begin{tabular}{|l|l|}
\hline
\textbf{Variable}  & \textbf{Description} \\ \hline
self.blocksize  &  Internal blocksize of LowMC in bits (also statesize)\\ \hline
self.blocksize\_bytes  &  Internal blocksize of LowMC in bytes\\ \hline
self.keysize  &  Lenght of private and public key in bits\\ \hline
self.rounds  &  Number of LowMC rounds\\ \hline
self.sboxes  &  Number of 3-Bit sboxes in LowMC\\ \hline
self.mpc\_rounds  &  Number of MPC rounds\\ \hline
self.hash\_length  &  Lenght of the SHAKE ouput as hashfunction in bits\\ \hline
self.lowmc  &  Name of the security level as string\\ \hline
\end{tabular}
\caption{Public variables}
\label{tab:secparams}
\end{center}
\end{table}
\noindent Before the private variables, we'll declare the implementation specific types (datastructures) and describe them in table \ref{tab:types}. They are defined in an own file, named \texttt{picnic\_types.py}. All the types are classes with only a constructor and some variables. Imagine them like C structs.\\
\begin{table}[htbp]
\begin{center}
\begin{tabular}{|l|l|l|}
\hline
\textbf{Class}  & \textbf{Variables} & \textbf{Description} \\ \hline
Publickey & self.public\_key & Public key as BitVector, length = self.keysize \\
		  & self.p			  & Plaintext message for public key generation as BitVector, \\
		  &					  & length =  self.keysize\\ \hline
View	  & self.i\_share     & Single input share as a BitVector, length = self.blocksize \\ 
		  & self.transcript  & Single transcript of a LowMC round as a BitVector,\\
		  &					  & length = AND gates in one LowMC round\\
		  & self.o\_share     & Single output share as BitVector, \\
		  &                   & length = self.blocksize \\ \hline
Commit-   & self.hash       & Hash of a single commitment as a bytearray,\\
ment      &                 & length = self.hash\_length \\
		  & self.n\_commitments & Not used by now \\ \hline
Proof     & self.seed\_1  & Seed of the first not revealed player in a round\\
		  & self.seed\_2  & Seed of the second not revealed player in a round\\
		  & self.i\_share & Input share of the revealed player in a round\\
		  & self.transcript & Transcript of the AND gates of the revealed player\\
		  & self.view\_3\_commit & A merged commit of the three players (to save space)\\ \hline
Signature & self.proofs & List of Proofs (class Proof), length = self.mpc\_rounds\\
		  & self.challenges & List of all challenges, length = self.mpc\_rounds\\
		  & self.salt & Salt as bytearray, length = self.blocksize\\ \hline
\end{tabular}
\caption{Picnic types (classes)}
\label{tab:types}
\end{center}
\end{table}\\
\noindent Now, with the knowledge about the types (classes) we can define the private variables. They mostly store the informations from the MPC rounds for every player (three of them). Therefore the views, commitments and seeds are two-dimensional arrays with the first dimension beeing the MPC rounds and the second dimension beeing the three players. Python does not have datatypes for arrays. They are nested lists, but can be accessed in the same way as arrays in other languages by bracket-notations ([dim 1][dim 2]). The private variables are described in table \ref{tab:privvars}.
\begin{table}[htbp]
\begin{center}
\begin{tabular}{|l|l|}
\hline
\textbf{Variable}  & \textbf{Description} \\ \hline
self.\_\_priv\_key    & Private key as BitVector, length = self.keysize\\ \hline
self.\_\_pub\_key     & Public key as type Publickey (class)\\ \hline  
self.\_\_views        & Two-dimensional array of Views (class)\\
			          & dims: [self.mpc\_rounds][3 (players)]\\ \hline
self.\_\_commitments  & Two-dimensional array of Commitments (class)\\
					  & dims: [self.mpc\_rounds][3 (players)] \\ \hline
self.\_\_seeds 		  & Two-dimensional array of seeds as BitVectors.\\
			          & Each BitVector has length = self.blocksize\\ 
			          & dims: [self.mpc\_rounds][3 (players)] \\ \hline
self.\_\_salt  		  & Salt as a BitVector, length = self.blocksize\\ \hline
self.\_\_tapes\_pos  & Position counter on the tapes. Type = Integer\\ \hline
self.\_\_challenges  & List of the challenges $e \in \{0,1,2\}$,\\
					  & length = self.mpc\_rounds \\ \hline
self.\_\_prove       & List of Proofs (class), length = self.mpc\_rounds\\ \hline
self.\_\_signature   & Stores the complete Signature (class),\\
					  & including proofs, challenges and salt \\ \hline
self.\_\_signature\_ser & Stores a serialized signature as bytearray\\ \hline
\end{tabular}
\caption{Private variables}
\label{tab:privvars}
\end{center}
\end{table}
\subsection{Verify a Picnic signature}
To verify a given signature, nearly the same steps as to create a signature are executed. The challenges decided which one of the three players in each round got revealed. The transcripts, seeds and  shares of the one revealed player got stored into the signature. In the verification the other two players must be recalculated. After recalculating the "missing" players data, the challenges are re-computed. \\ \\
In figure \ref{fig:mainloops} can be seen, that the computation of the challenges is the secondlast step in the creation of the signature. The data from all three players are merged into a single hash and the challenges are derived from this hash. Only after the challenges are fixed, the proofs for the signature are assembled, based on the challenges. In other words the challenges depend on the calculated data of all three players.\\ \\
\textbf{The signature is accepted as valid,} if the re-calculation of the "missing" (not revealed) players and the data from the revealed player computes to the same challenges as in the signature.\\ \\
The main algorithmic differences between sign and verify are that each MPC round and each LowMC round only calculates two players in the verification. And which two players are calculated depends on the round challenge in the signature. Therefore all internal functions are re-written to accept the round challenge as an additional parameter and calculate only the two "missing" players. This seems a little redundant in the sourcecode but keeps good review and learning capabilities. The verify functions can be recognized by the postfix \texttt{\_verify} in their names.  
\subsection{The API to Python-Picnic}
In comparison to \texttt{Python-LowMC} there are no private functions in \texttt{Python-Picnic}. All functions follow the Python scheme for public functions (no leading \texttt{\_\_}). That could be changed in future reworks. Therefore the functions to use \texttt{Python-Picnic} are described here. Additionaly there ist a testfile (\texttt{tests.py}) in which the usage can be seen by function-call examples.\\ \\
\texttt{Picnic-LowMC} got expanded from the standalone version (in the LowMC Github) to a version for the Picnic needs. For example the sbox-function in LowMC needs to store the outcomes of the AND-gates for the signature. That was not a part in the standalone LowMC. A new, expanded version of LowMC is included within \texttt{Python-Picnic} as the file \texttt{lowmc.py}.\\ \\
The Picnic specifications inculde a scheme for serializing signatures. The serialization and file-storage of a signature are implemented in \texttt{Python-Picnic}. That enables a way to compare signatures to the ones from the reference implementation. The file-storage writes textfiles with the serialized signature as a HEX-string. This is not the most compact way, but enables human readability and comparison.\\ \\
The serialization of the keys is also a part of the Picnic specification but is not implemented in \texttt{Python-Picnic} so far. This is a priority issue for future works. That means, that the keys are gone, if the Picnic-object is destroyed (the execution of the sourcecode ends). A function for setting the private key and the plaintext message is included. This function then derives the public key.
\subsubsection{\texttt{generate\_keys(p, priv\_key)}}
Sets or generates the plaintext message $p$ (for LowMC key generation) and the private key $pk$. If the parameters are given, they are setted. If left emtpy, they are generated with \texttt{os.urandom()}. The public key then gets derived via LowMC encrpytion. The parameters must be bytearrays with correct length according to the security parameter. So far only "Picnic-L1" is implemented. Therefore the sizes must be 16 bytes. The data is stored in the matching private variables from table \ref{tab:privvars}.
\subsubsection{\texttt{sign(message)}}
Signs a message $M$. The keys must be set or generated before. The message must be passed as a bytearray of arbitrary length in the bounds $1 \leq \vert M \vert \leq 2^{55}$. The signature gets stored into the private variable \texttt{self.\_\_signature} (see table \ref{tab:privvars}). 
\subsubsection{\texttt{verify(message)}}
Verifies a signature regarding to the message. The keys must be set or generated before. The message must be passed as a bytearray of arbitrary length in the bounds $1 \leq \vert M \vert \leq 2^{55}$. The signature must be available in the private variable \texttt{self.\_\_signature} before calling \texttt{verify}. This can be done by signing a message or by reading one from a file before.
\subsubsection{\texttt{serialize\_signature()}}
Serializes a signature from the private variable \texttt{self.\_\_signature} and stores it into the private variable \texttt{self.\_\_signature\_ser}.
\subsubsection{\texttt{deserialize\_signature()}}
Deserializes a signature from the private variable \texttt{self.\_\_signature\_ser} and stores it into the private variable \texttt{self.\_\_signature}.
\subsubsection{\texttt{write\_ser\_sig\_to\_file(filename)}}
Writes a serialized signature from the private variable \texttt{self.\_\_signature\_ser} to a textfile. The filename must be provided as a string. The textfile then contains the serialized signature as a HEX-string.
\subsubsection{\texttt{read\_ser\_sig\_to\_file(filename)}}
Reads a textfile with the given filename (string). The textfile must contain a serialized signature as HEX-string. The serialized signature then gets stored into the private variable \\ \texttt{self.\_\_signature\_ser}.
\subsubsection{\texttt{print\_signature()}}
Does a printout into the console of the signature in the private variable \texttt{self.\_\_signature}. The variable must contain a signature.
\subsubsection{\texttt{print\_signature\_ser()}}
Does a printout into the console of the serialized signature in the private variable \\ \texttt{self.\_\_signature\_ser} as a HEX dump. The variable must contain a serialized signature.
\subsection{Prerequisites}
The code is tested with Ubuntu 16.04 LTS and Python3.6. The package "BitVector" for python is required. The package "hashlib" is used for the SHA3-SHAKE parts. SHAKE is available in "hashlib" since Python 3.6. Therefore \texttt{Python-Picnic} will not run with Python versions lower than 3.6. It is recommended to use a virtual environment for Python, like \texttt{virtualenv}. In (very) short lines:\\ \\
\texttt{virtualenv --p /usr/bin/python3.6 myvenv}\\
\texttt{source /path\_to\_myvenv/bin/activate}\\
\texttt{<myvenv>pip install BitVector}\\
\texttt{<myvenv>python tests.py}\\
%\bibliographystyle{unsrt}
%\bibliography{}

\end{document}
